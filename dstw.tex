%% ****** Start of file apstemplate.tex ****** %
%%
%%
%%   This file is part of the APS files in the REVTeX 4.2 distribution.
%%   Version 4.2a of REVTeX, January, 2015
%%
%%
%%   Copyright (c) 2015 The American Physical Society.
%%
%%   See the REVTeX 4 README file for restrictions and more information.
%%
%
% This is a template for producing manuscripts for use with REVTEX 4.2
% Copy this file to another name and then work on that file.
% That way, you always have this original template file to use.
%
% Group addresses by affiliation; use superscriptaddress for long
% author lists, or if there are many overlapping affiliations.
% For Phys. Rev. appearance, change preprint to twocolumn.
% Choose pra, prb, prc, prd, pre, prl, prstab, prstper, or rmp for journal
%  Add 'draft' option to mark overfull boxes with black boxes
%  Add 'showkeys' option to make keywords appear
\documentclass[aps,pre,preprint,groupedaddress,showkeys]{revtex4-2}
%\documentclass[aps,prl,preprint,superscriptaddress]{revtex4-2}
%\documentclass[aps,prl,reprint,groupedaddress]{revtex4-2}

% You should use BibTeX and apsrev.bst for references
% Choosing a journal automatically selects the correct APS
% BibTeX style file (bst file), so only uncomment the line
% below if necessary.
%\bibliographystyle{apsrev4-2}

\draft % marks overfull lines with a black rule on the right

\usepackage{graphicx}% Include figure files

\usepackage[mathlines]{lineno}% Enable numbering of text and display math
\modulolinenumbers[1]% Line numbers with a gap of 5 lines
\linenumbers\relax % Commence numbering lines

\usepackage{siunitx}


\begin{document}
% Use the \preprint command to place your local institutional report
% number in the upper righthand corner of the title page in preprint mode.
% Multiple \preprint commands are allowed.
% Use the 'preprintnumbers' class option to override journal defaults
% to display numbers if necessary

%Title of paper
\title{Disturbed cycles of sawtooth wave-like pressure changes in a pipe-chamber two-phase flow system.}

% repeat the \author .. \affiliation  etc. as needed
% \email, \thanks, \homepage, \altaffiliation all apply to the current
% author. Explanatory text should go in the []'s, actual e-mail
% address or url should go in the {}'s for \email and \homepage.
% Please use the appropriate macro foreach each type of information

% \affiliation command applies to all authors since the last
% \affiliation command. The \affiliation command should follow the
% other information
% \affiliation can be followed by \email, \homepage, \thanks as well.
\author{Yo Kanno}
\email{kanno@email.address}

\author{Mie Ichihara}
\email{ichihara@eri.u-tokyo.ac.jp}

%\email[]{Your e-mail address}
%\homepage[]{Your web page}
%\thanks{}
%\altaffiliation{}
\affiliation{Earthquake Research Institute, University ofTokyo, Yayoi, 1-1-1, Bunkyo, Tokyo 113-0032, Japan}

%Collaboration name if desired (requires use of superscriptaddress
%option in \documentclass). \noaffiliation is required (may also be
%used with the \author command).
%\collaboration can be followed by \email, \homepage, \thanks as well.
%\collaboration{}
%\noaffiliation

\date{\today}
\begin{abstract}
% insert abstract here
We study experimentally sawtooth wave-like pressure changes associated with gas-liquid flow in a pipe-chamber system.
Due to the coupling of an elastic capacitance of a gas chamber and non-linear pressure drop in the pipe, the characteristic pressure waveforms are observed.
For specific conditions, pressure cycles are disturbed spontaneously even at a constant flow rate and chamber volume.
We report a modulation of the pressure waveform through experimental parameters.  
Through image analyses, we point out the effect of surface disturbance of downward flow.
\end{abstract}

% insert suggested keywords - APS authors don't need to do this
\keywords{Laboratory experiment, Sawtooth waveform, Dynamical system, Flow-induced oscillation}

%\maketitle must follow title, authors, abstract, and keywords
\maketitle

% body of paper here - Use proper section commands
% References should be done using the \cite, \ref, and \label commands
\section{Introduction}\label{intro}
Sawtooth wave-like signals can be found in nature, everyday's life to large-scale natural phenomena. 
For instance, the sawtooth waves are the most common waveforms used to create sounds with music synthesizers.


\section{Experimental method}\label{met}
\subsection{Experimental setup}\label{setup}
The experimental system comprised a chamber and pipe, where a constant gas flux ($Q_\mathrm{in}$) was maintained in the chamber, and the gas-liquid flow in the pipe. 
The chamber, pipe, and connecting-joint were constructed using acrylic glass to ensure rigidity and enable direct observations. 
The compressible gas volume in the chamber ($V_\mathrm{c}$) is maintained at a constant value during each experimental run.
$Q_\mathrm{in}$ was regulated by a microneedle valve (IBS FMNV2) that ensured a constant gas flux of between \SI{1}{\m^3 / \s}. 
We used sugar syrup diluted in distilled water as the liquid phase. 
The liquid behaved as a Newtonian fluid, and we measured its viscosity using a BOHLIN CVO Rheometer and adjusted it to \SI{1}{\Pa \s} at the experimental temperature of \SI{25}{\degreeCelsius}. 
Injected gas flowed through the pipe and forced the liquid residing in the pipe upward, resulting in gas-liquid flow within the pipe.
A pressure sensor (KISTLER 701A with a 5011A charge amplifier) was installed to measure pressure changes in the chamber, and a microphone (Br\"uel-Kj\ae r 4193+2669L with a Nexus 2690 signal conditioner) was mounted to the top of the pipe. Data were sampled at \SI{10}{kHz} using a PC-based data acquisition system (DEWETRON DEWE-211).
A high-speed black-and-white camera (Photoron FASTCAM Mini) was synchronized with the data acquisition PC and was focused on the lowermost \SI{200}{\mm} of the pipe, where gas-liquid flow occurred.
The field of view was illuminated by a flat backlight source. We used a pale-color syrup to improve the visibility of the liquid phase against the gas phase and the pipe. 
Dark areas were thus shaded according to the thickness of the liquid across the pipe. 
We used the resulting distribution of light intensity to observe flow patterns.
The controlled experimental parameters were $V_\mathrm{c}$, $Q_\mathrm{in}$, and the amount of syrup in the pipe; as defined by the initial syrup height $H_\mathrm{s}$. 
These parameters were fixed during each experimental run. 
We conducted 143 runs in total, during which the parameters varied from m3/s for $Q_\mathrm{in}$, 20 to \SI{157} {\cm^3} for $V_\mathrm{c}$, and 60 to \SI{150}{\mm} for $H_\mathrm{s}$.

\subsection{Image Analyses method}\label{ime}
The thickness of the downward film flow and the position of the liquid slug were analyzed using image analysis.

%Fig. \ref{Rbottom} に, ある時間, ある高さにおけるパイプ画像水平方向断面の暗度分布を示す.  
Fig. \ref{Rbottom} shows the darkness distribution of the horizontal cross section of a pipe image at a certain height for a certain time.
%パイプ中心に気体が存在する部分 (ガススラグ, 詳細は \S \ref{flowinpipe}) では, パイプ領域内に二つピークが存在し, 管内壁の液膜 (液膜流, 詳細は \S \ref{flowinpipe}) の厚みが薄くなるほどこの二つのピーク間の幅は大きくなる. 
In the part where gas exists at the center of the pipe, there are two peaks in the pipe area, and as the thickness of the liquid film on the inner wall of the pipe becomes thinner, the width between these two peaks becomes larger.
%空気, 水あめ, アクリルの屈折率をそれぞれ1, 1.44 (島津製作所, \url{https://www.shimadzu.co.jp/opt/products/ref/ref-app05.html}), 1.49 (島津製作所, \url{https://www.shimadzu.co.jp/opt/products/ref/ref-app03.html}) とした上で, 空気, アクリル部分, 及び色付けした水あめ領域を通過する光路長に比例して光が吸収され, 暗くなると考えた. 
Assuming that the refractive indices of air, liquid, and acrylics are 1, 1.44, and 1.49, respectively, it was thought that light was absorbed and darkened in proportion to the optical path length passing through the air, the liquid, and the acrylic part.
%カメラへ到達した光が, どのような光路で空気, アクリルパイプ, 液部分を通過して, バックライトの平面光源から到達したのかを, スネルの法則を用いて見積もると, Fig. \ref{raytraseDk} のようになる. 
Using Snell's law, the light paths from the flat backlighting source reaching the camera has passed through the air, the acrylic pipe and the liquid parts are calculated as shown in Fig. \ref{raytraseDk}.
%ただし, 光が空気及びアクリル部分を通過する際にも, 距離に比例して光が吸収されるとし, 水あめ領域を通過するときには, 空気及びアクリル領域に比べて光を3倍吸収すると考えた.
When light passed through the air and acrylic part, it was assumed that the light is absorbed in proportion to the distance, and when passing through the liquid parts, it was considered to absorb three times the light compared to the air and acrylic area.
%仮定する吸収率によって暗度分布が多少異なるが, 暗度のピーク位置には影響を及ぼさない.
Although the darkness distribution differs somewhat depending on the assumed absorption factor, it does not affect the peak positions of the darkness distributions.
%実験画像と同様に, パイプ中心に気体がある場合は二つの暗度ピークがみられ, 液膜の厚さが小さくなるほど, ピーク間の幅は大きくなる. 
As a result of the calculation, as in the experimental results, when there was gas part at the pipe center, two darkness peaks are seen, and the smaller the thickness of the liquid film, the larger the width between the peaks.
%そこで, 暗度ピーク間の幅とパイプ内径の比を実験画像と計算で比較し, パイプ各高さで液膜流厚みを推定する. 
%一方, 明瞭な暗度のピークが見られない断面は, 液体で満たされた部分 (液スラグ, 詳細は \S \ref{flowinpipe})と判定する.
Therefore, the width between the darkness peaks and the ratio of the inner and outer diameter of the pipe are compared with the experimental image and calculation to estimate the actual thickness of the downward film flow at each height of the pipe.
On the other hand, the cross section where no clear peak of the darkness was judged to be the part filled with liquid.

\subsection{Flow image}\label{flowim}
%撮影したビデオ映像の各ビデオフレームにおいて, 高さ方向の液膜流厚み分布を推定する. 
The thickness distribution of downward film flow in the height direction was estimated at each video frame.
%この結果を幅 1ピクセルの画像として, 時間順に水平方向に連結し, 流動画像を作成する. 
These results as a 1-pixel-wide image were connected horizontally in time order to create a flow image.
%流動画像の例を Fig. \ref{UnimslugimageAn}a, \ref{UnimimVSpre} に示す (詳細は \S \ref{PandF}で後述). 
An example of the flow image is shown in Fig. \ref{UnimslugimageAn}a, \ref{UnimimVSpre}. 
%赤い部分は, 暗度ピークが不明瞭な領域であり, 液スラグ部分に対応している. 
The red parts are the area where the dark peaks are unclear and correspond to the ascending liquid slug part while orange to blue part represent downward film flow.
%また, オレンジ色から青色の領域は, 液膜流の厚みを表しており, 液膜流は流下しているため, 全体に右下がりの陰になっている. 

%作成した流動画像から, 閾値処理によって画像の二値化を行い, 液スラグ領域のピクセルに 1を, それ以外に 0を割り当てる. 
The created flow image was binarized by thresholding, and 1 was assigned to pixels in the liquid slug area and 0 was assigned to others.
%ある液スラグ領域のピクセルに対して, 上下左右の近傍8点のうち, どれかが 1であれば, 同じ液スラグ領域であると判定する, 8点近傍連結を用いて, それぞれの液スラグごとにその画像領域をラベリングし, それぞれの液スラグの挙動を追跡した. 
For any pixel in a liquid slug area, if any one of the eight points in the upper, lower, left, and right areas is 1, it was determined that there is the liquid slug area was the same. 
The image area was labeled and the behavior of each liquid slug was traced.
%二値化および領域のラベリング処理には, MATLAB の Image Processing Toolbox (imagesegmenter) を用いた. 
MATLAB's Image Processing Toolbox (images segmenter) were used for binarization and region labeling.


\section{Result}\label{res}
\subsection{Characteristic pressure changes in the chamber}

%パラメータ $Q_\mathrm{in}$, $V_\mathrm{c}$ の値によって計測された様々な特徴的圧力波形と, パイプ内流れの詳細な解析結果を示す. 
The various characteristic pressure waveforms measured by the values of the parameters $Q_\mathrm{in}$ and $V_ \mathrm{c}$ and detailed analysis results of the flow in the pipe are shown in following part.

\subsubsection{Unimodal STW}
%Fig. \ref{ResultPhaseD}内, $\diamondsuit$ で示した実験パラメータにおいて, 一定の時間幅を持ったノコギリ波状の圧力変動が発生した (Fig. \ref{PplotEx}a).
In the experimental parameters shown as $\diamondsuit$  in Fig. \ref{ResultPhaseD}, sawtooth wave-like pressure change with a fixed time interval was generated (Fig. \ref{PplotEx}a).
%本研究では, ゆっくりとしたチャンバー圧力の増圧に続いて, 圧力が -5 kPa/s より速く減少し, かつ, 後述する流動様式の遷移が発生するような急減圧が発生した場合, STWが発生したと定義する. 
In this study, the term of STW will be used if the pressure decreases more rapidly than -5 kPa/s following the gradual pressure increase, which causes the flow mode transition described later.
%チャンバー圧力波形の, 急減圧発生から次の急減圧までの時間 ($\Delta t$) をSTW時間幅とし,そのヒストグラムを Fig. \ref{histogramall}a に示す. 
The time ($\Delta t$) from the onset of the abrupt pressure drop to the next pressure drop of the chamber pressure waveform was taken as the STW time width and the histogram is shown in Fig. \ref {histogramall}a.
%この時, STW時間幅のヒストグラムは Unimodal な分布となることから, この圧力波形を Unimodal STWとする. Unimodal STW は, \cite{kanno2018} で報告した Periodic STWに対応している.
Let this pressure waveform be the unimodal STW since the histogram of the STW time width became a unimodal distribution.
The Unimodal STW corresponds to the Periodic STW reported at previous study \citep{kanno2018}.
%本研究では, 過去の実験に比べて, より大きい $Q_\mathrm{in}$ まで実験を行なった結果, 後述するように, STW波形により多くのバリエーションを見出したため, 特徴的波形の分類を細分化している.
The classification of characteristic waveforms was subdivided since more variations than the previous study were found in the STW waveform as a result of experimenting to larger $Q_ \mathrm{in}$ than past experiments.
%パイプ内流れとの対応関係は, \S \ref{PandF} でまとめて示す.
The correspondence with the flow pattern in the pipe will be summarized in \S \ref{PandF}.

\subsubsection{Small Fluctuation}
Unimodal STW の発生領域よりもより小さい $Q_\mathrm{in}$, $V_\mathrm{c}$ 下では (Fig. \ref{ResultPhaseD}, ×), STWの発生が見られず, 圧力の Small fluctuation が発生した (Fig. \ref{PplotEx}b)  これは, \cite{kanno2018} で報告した Non-STW に対応している. 圧力波形のスペクトルにおけるピーク周波数から推定した振動周期は 5 - 10 秒程度で, STW 発生時よりもやや短く, 振幅が小さいことが特徴である \cite[Fig. 4]{kanno2018}.

\subsubsection{Disturbed STW}
Unimodal STW 発生領域よりも大きい $Q_\mathrm{in}$ の条件(Fig. \ref{ResultPhaseD}, ◯) では, 圧力波形は STWの特徴を持っているものの, 周期が一定ではない圧力振動パターンが発生する (Fig. \ref{PplotEx}c). この時の STW時間幅ヒストグラムは, Fig. \ref{histogramall}c で示すように, Unimodal STWと比べてばらつきを持って分布していることから, この特徴的圧力波形を, Disturbed STWとする.
$Q_\mathrm{in}$ がやや大きい領域 (Fig. \ref{ResultPhaseD}, ☆) では, Unimodal パターン (Fig. \ref{PplotEx}d, $t=0\sim10$, $45\sim85$ s) と Bimodal パターン (Fig. \ref{PplotEx}d, $t=10\sim45$, $85\sim100$ s)を持つ STWサイクルが交互に出現する (Fig. \ref{PplotEx}d). ヒストグラムは, Trimodal な分布になっているため, 以下, この圧力波形を Trimodal STWとする (Fig. \ref{histogramall}e). それぞれのピークは Unimodal STWに比較してばらつきを持って分布しているため, Trimodal STWも Disturbed STWの一種であると分類する. Disturbed STW発生時の波形は, $\Delta t$ が大きいほど, STW各サイクルの圧力振幅 ($\Delta P_c$) が大きくなっている (Fig. \ref{histogramall}d, f).

\subsubsection{Unimodal-HQ STW, n-Type cycle}
Disturbed STWが発生する領域よりもさらに $Q_\mathrm{in}$ を大きくすると, 再び Unimodal STWが発生する (Fig. \ref{ResultPhaseD}, ▽, \ref{PplotEx}e). この時の特徴的波形を Unimodal-HQ STWとする. また, $V_\mathrm{c}$ が小さいが, $Q_\mathrm{in}$ が Disturbed STWと同程度の条件 (Fig. \ref{ResultPhaseD}, △) では, 急激な圧力現象が一定周期で発生するが, STWに特徴的なゆっくりとした増圧過程は見られず, "n" のような形をしたチャンバー圧力サイクルが繰り返される (Fig. \ref{PplotEx}f). このパラメータ下の振動を, n-Type cycle と呼び, Unimodal STWとは区別する. 




\subsection{Flow characteristic}
%作成した流動画像に対して, 画像解析を用いて液スラグ部分を抽出することによって, 各液スラグの位置, 速度, 長さを推定する (\S \ref{ime}). 
The position, velocity, and length of each liquid slug were estimated by extracting the liquid slug part using image analyses on the created flow image (\S \ref{ime}).
%この解析結果を用いて, 液スラグ再生成位置と, 液スラグ速度-液膜流厚みの関係を解析する.
Using these results, the relationship between the liquid slug reconstruction position and the liquid slug velocity-downward film flow thickness were analyzed.

%Unimodal STWや Small fluctuation 発生時には, 液スラグ再生成はパイプ下端で, ほぼ一定間隔で発生していたが, Disturbed STWでは, 液膜流領域内でも液スラグ再生成が発生していた.
Under the unimodal STW and small fluctuation conditions, liquid slug reconstruction was generated at a substantially constant interval at the lower end of the pipe, but in Disturbed STW conditions, liquid slug reconstruction was also generated in the downward film flow region.
%液スラグの再生成位置が, 特徴的圧力波形の違いによってどのように分布しているかを, Fig. \ref{Repsite} に示す. 
Fig. \ref{Repsite} shows how the reconstruction positions of the liquid slug were distributed due to the difference in the characteristic pressure waveform.
%Unimodal STW 発生時には, ほぼ全ての液スラグが決まった位置 (パイプ下端) で再生成する. 一方で, Disturbed STWは, スラグ再生成の位置にばらつきがある. 
Under the Unimodal STW conditions, almost all the liquid slug was reconstructed at a fixed position: pipe bottom.
On the other hand, there was variation in the position of slug reconstruction position under the Disturbed STW conditions (hereafter termed halfway reconstruction).
%パイプの途中で液スラグが再生成するとき, その原因となった液膜流擾乱は, 比較的液膜流厚みが大きい擾乱になっている (Fig. \ref{DistimVSpre}, \ref{TrimimVSpre}). 
Surface disturbance of downward film flow was relatively large when halfway reconstruction occurred.
%また, 流動画像によれば, 液スラグの速度が速い時には, その直下の液膜流厚みが厚く, 液スラグがゆっくり動くときには薄くなっている. この関係を画像解析を用いて定量的に抽出する.
Also, according to the flow image, when the velocity of the liquid slag is large, the thickness of the downward film flow immediately below it was thick, and it becomes thin when the liquid slug moving slowly.
This relationship was extracted quantitatively using image analyses.

%映像を取得した全ての実験結果に対して, 液スラグ下端の移動速度と, 液スラグ下端直下の液膜流厚みの関係を Fig. \ref{RDKall} に示す. 以下, 液膜流厚みを $h_f$ とする.
The relationship between the moving velocity at the lower end of the liquid slug and the thickness of the downward film flow immediately below the lower end of the liquid slug is shown in Fig.  for all the experimental results for which flow images were acquired.
In the following, let the thickness of liquid film flow be $h_f$.
%この結果から, 液スラグの移動速度が小さい (大きい) と, その直下の$h_f$が小さい (大きい) ことがわかる.
From this result, it was obtained that $h_f$ immediately below the liquid slug was small (large) if the moving speed is small (large).
%この結果によれば, 速度が大きくなり, 液スラグの直上の$h_f$よりも, 直下の$h_f$が大きくなれば, 液スラグに流入する液相よりも排出される液相が多くなると考えられる. 
When the velocity increased and $h_f$ below the liquid slug became larger than $h_f$ above the liquid slug, it is thought that the amount of liquid phase discharged is greater than the liquid phase flowing into the liquid slug.
%すなわち, 液スラグが液膜流上端に達する前に, 液スラグ長さが小さくなっていくことが示唆される.
That is, it is suggested that the length of the liquid slug becomes smaller before the liquid slug reaches the upper end of the downward film flow.


%それぞれの特徴的圧力波形において, 液スラグ直下の液膜流厚み頻度分布計算した結果を Fig. \ref{RDKhist} に示す. 
For each characteristic pressure waveform, Fig. \ref{RDKhist} shows the calculated results of the thickness distribution of the downward film flow immediately below the liquid slug.
%この結果, Uniodal STWでは, 液膜流の厚みがほぼ均一である一方で, Distubed STWでは液膜流の厚み分布にばらつきがあり, より厚い液膜流擾乱が発生している.
As a result, under the Unimodal STW conditions, the thickness of the liquid film flow is almost uniform, but in the Disturbed STW conditions, the thickness distribution of the downward film flow is uneven, and thicker surface disturbance occurs.



\section{Mechanics}\label{mec}
\subsection{STW waveforms}
The generation mechanism of STW waveform and its generation condition has been examined based on a simple mathematical model.


\subsection{Effective viscosity changes}

\section{Discussion}\label{mec}

\section{Conclusion}\label{con}


\begin{eqnarray}
a+b=c
\label{eq:one}.
\end{eqnarray}
Note the open one in Eq.~(\ref{eq:one}).

\begin{figure}
\includegraphics{fig_1}% Here is how to import EPS art
\caption{\label{fig:epsart} A figure caption. The figure captions are automatically numbered.}
\end{figure}


% If in two-column mode, this environment will change to single-column
% format so that long equations can be displayed. Use
% sparingly.
%\begin{widetext}
% put long equation here
%\end{widetext}

% figures should be put into the text as floats.
% Use the graphics or graphicx packages (distributed with LaTeX2e)
% and the \includegraphics macro defined in those packages.
% See the LaTeX Graphics Companion by Michel Goosens, Sebastian Rahtz,
% and Frank Mittelbach for instance.
%
% Here is an example of the general form of a figure:
% Fill in the caption in the braces of the \caption{} command. Put the label
% that you will use with \ref{} command in the braces of the \label{} command.
% Use the figure* environment if the figure should span across the
% entire page. There is no need to do explicit centering.

% \begin{figure}
% \includegraphics{}%
% \caption{\label{}}
% \end{figure}

% Surround figure environment with turnpage environment for landscape
% figure
% \begin{turnpage}
% \begin{figure}
% \includegraphics{}%
% \caption{\label{}}
% \end{figure}
% \end{turnpage}

% tables should appear as floats within the text
%
% Here is an example of the general form of a table:
% Fill in the caption in the braces of the \caption{} command. Put the label
% that you will use with \ref{} command in the braces of the \label{} command.
% Insert the column specifiers (l, r, c, d, etc.) in the empty braces of the
% \begin{tabular}{} command.
% The ruledtabular enviroment adds doubled rules to table and sets a
% reasonable default table settings.
% Use the table* environment to get a full-width table in two-column
% Add \usepackage{longtable} and the longtable (or longtable*}
% environment for nicely formatted long tables. Or use the the [H]
% placement option to break a long table (with less control than 
% in longtable).
% \begin{table}%[H] add [H] placement to break table across pages
% \caption{\label{}}
% \begin{ruledtabular}
% \begin{tabular}{}
% Lines of table here ending with \\
% \end{tabular}
% \end{ruledtabular}
% \end{table}

% Surround table environment with turnpage environment for landscape
% table
% \begin{turnpage}
% \begin{table}
% \caption{\label{}}
% \begin{ruledtabular}
% \begin{tabular}{}
% \end{tabular}
% \end{ruledtabular}
% \end{table}
% \end{turnpage}

% Specify following sections are appendices. Use \appendix* if there
% only one appendix.
\appendix
\section{Here is an appendix}

% If you have acknowledgments, this puts in the proper section head.
\begin{acknowledgments}
 put your acknowledgments here, Thank you!
\end{acknowledgments}

% Create the reference section using BibTeX:
\bibliography{basename of .bib file}

\end{document}
%
% ****** End of file apstemplate.tex ******

