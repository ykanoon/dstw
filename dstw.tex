\documentclass[12pt]{article}
\usepackage[top=30truemm, bottom=30truemm]{geometry}
%draftの場合[dvipdfmx]を[draft]
\usepackage[dvipdfmx]{graphicx}
\usepackage{newtxtext}
\usepackage{layout}
\usepackage{lineno}

\usepackage{url}
\usepackage{float}
\usepackage{array}
\usepackage{subcaption}
\usepackage{setspace} 
\setstretch{2}
\usepackage{makeidx}
%bibliographyに必要なパッケージ natbibzip内
%https://ja.sharelatex.com/learn/Bibliography_management_with_natbib
\usepackage{natbib}
\usepackage{lscape}
\usepackage[dvipdfmx]{color}
\usepackage[dvipdfmx]{hyperref}
\usepackage{pxjahyper} %%hyperref読み込みの直後に
\usepackage[toc,title,page]{appendix}

\begin{document}
\linenumbers
\begin{abstract}
Studying the physics of experiments using common materials to simulate geological processes frequently provides novel ideas and insight that can be applied to natural phenomena.
In this study, a physical system of a laboratory experiment that is assumed to be mathematically equivalent to volcanic eruption systems and its behaviors and physical processes are investigated.
This study aims to obtain through a laboratory experiment such new insight and ideas that it would be difficult to find directly from field observation and mathematical modeling of volcanic systems.
\end{abstract} 

\newpage
\cleardoublepage

\renewcommand{\thesection}{\Roman{section}.}
\renewcommand{\thesubsection}{\Alph{subsection}.}

\section{Introduction}\label{intro}
It is well-established that volcanoes are sources of infrasound, defined as acoustic, or sound, waves below 20 Hz.
\cite{kanno2018} did experiments.

\section{Experimental setup}\label{setup}

\section{Result}\label{res}
\subsection{Signals at bursting}

\section{Mechanics}\label{mec}

\section{Discussion}\label{mec}

\section{Conclusion}\label{con}



%bibliography作成にはMendeleyを用いた
%Mendeleyを用いてbibファイルをこのtexファイルと同じディレクトリ内に作成. ターミナル上で " pbibtex ファイル名" を実行してコンパイル
%http://pioneerboy.hatenablog.com/entry/2014/01/18/214446
%http://ftp.yz.yamagata-u.ac.jp/pub/CTAN/macros/latex/contrib/natbib/natbib.pdf
%Agu08.bst をどこから入手したのかわすれてしまった...
%agu_template内
\clearpage \newpage
\bibliographystyle{agu08}
\bibliography{library} 
\addcontentsline{toc}{section}{References}


\end{document}